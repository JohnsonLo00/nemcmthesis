
\section{模型的建立与求解}

% -------------------- 问题 1 -------------------- %

\subsection{对问题1的求解}

\zhlipsum[1]

\begin{equation}\label{equ:麦克斯韦方程组}
\begin{cases}
\oint_{\boldsymbol{S}}\boldsymbol{D}\cdot\diff\boldsymbol{S} &= q_0, \\
\oint_{\boldsymbol{S}}\boldsymbol{B}\cdot\diff\boldsymbol{S} &= 0, \\
\oint_{\boldsymbol{l}}\boldsymbol{E}\cdot\diff\boldsymbol{l} &= -\iint_{\boldsymbol{S}}\frac{\partial \boldsymbol{B}}{\partial t}\cdot\diff\boldsymbol{S}, \\
\oint_{\boldsymbol{l}}\boldsymbol{H}\cdot\diff\boldsymbol{l} &= \boldsymbol{I}_0 + \iint_{\boldsymbol{S}}\frac{\partial \boldsymbol{D}}{\partial t}\cdot\diff\boldsymbol{S}
\end{cases}
\end{equation}

\zhlipsum[2]

\subsubsection{结果求解与分析}

%\fourthsection{四级标题名称} %必要情况下可能会使用四级标题

% -------------------- 问题 2 -------------------- %

\subsection{对问题2的求解}

\begin{gather}\label{equ:规划模型}
\min\quad{f(\boldsymbol{x})} \\
\text{s.t.}\quad\begin{cases}
    a \leq x_i\leq b &, j=1,2,\cdots ,\alpha \\
	\max\limits_{j=1,2,\cdots ,\alpha}{\{A_{i,j}\}} \leq 24 &, i=1,2,\cdots ,\beta_1\\
	\max\limits_{j=1,2,\cdots ,\alpha}{\{B_{i,j}\}} \leqslant 24 &, i=\beta_1, \beta_1+1,\cdots, \beta_2 \\
    \end{cases}
\end{gather}


\subsubsection{结果求解与分析}

% -------------------- 问题 3 -------------------- %

\subsection{对问题3的求解}

\zhlipsum[3]

\begin{equation}\label{equ:VandermondeDet}
D_n = \begin{pmatrix}
1 & 1 & \ldots & 1 \\
x_1 & x_2 & \ldots & x_n \\
\vdots & \vdots & & \vdots \\
x_1^{n-1} & x_2^{n-1} & \ldots & x_n^{n-1}
\end{pmatrix} = \prod_{1\leq j<i\leq n}{(x_i-x_j)}.
\end{equation}

\zhlipsum[4]

\subsubsection{结果求解与分析}

% -------------------- 问题 4 -------------------- %

\subsection{对问题4的求解}

伪代码的使用示例如下所示。

\begin{algorithm}[!ht]
\SetKwData{Left}{left}\SetKwData{This}{this}\SetKwData{Up}{up} % 定义快捷变量,后续使用时只需:反斜杠+变量名
\SetKwFunction{Union}{Union}\SetKwFunction{FindCompress}{FindCompress} % 定义函数,后续使用时只需:反斜杠+函数名
\SetKwInOut{Input}{输入}\SetKwInOut{Output}{输出} %此处可自定义输入、输入的名称格式

在此处添加不带编号的内容(若无,则将该行注释即可)。\\
\LinesNumbered % 使下列算法描述带行号
\Input{A bitmap $Im$ of size $w\times l$} % 输入
\Output{A partition of the bitmap} % 输出
\BlankLine

\emph{special treatment of the first line}\;
\For{$i\leftarrow 2$ \KwTo $l$}{
  \emph{special treatment of the first element of line $i$}\;
  \For{$j\leftarrow 2$ \KwTo $w$}{\label{forins}
    \Left$\leftarrow$ \FindCompress{$Im[i,j-1]$}\;
    \Up$\leftarrow$ \FindCompress{$Im[i-1,]$}\;
    \This$\leftarrow$ \FindCompress{$Im[i,j]$}\;
    \If( \tcp*[h]{此处添加注释:O(\Left,\This)==1} ){\Left compatible with \This}{\label{lt} % \tcp*[h]{...}表示非对齐的注释
      \lIf{\Left $<$ \This}{\Union{\Left,\This}} %含else...的if语句用命令\lIf
      \lElse{\Union{\This,\Left}}
    }
    \If(\tcp*[f]{O(\Up,\This)==1}){\Up compatible with \This}{\label{ut} % \tcp*[f]{...}表示居右对齐的注释
      \lIf(\tcp*[f]{再次用Union函数}){\Up $<$ \This}{\Union{\Up,\This}}
      \tcp{\This is put under \Up to keep tree as flat as possible}\label{cmt} % \tcp{...}表示行内的注释
      \lElse{\Union{\This,\Up}}\tcp*[h]{\This linked to \Up}\label{lelse}
    }
  }
  \lForEach{element $e$ of the line $i$}{\FindCompress{p}}
}
\caption{不相交分解(disjoint decomposition)}\label{pcode: 不相交分解} %伪代码对应算法的标题及其引用标签
\end{algorithm}

引用时的格式为:算法\ref{pcode: 不相交分解}实现了...。

% -------------------- 结果求解与分析 -------------------- %

\subsubsection{结果求解与分析}



