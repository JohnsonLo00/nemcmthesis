
\begin{appendices}

\section{相关定理的证明}

\zhlipsum*[2]

\section{程序代码}

%以文件形式插入代码
\lstinputlisting[ language=C++, title={\raggedright\normalsize 计算$\boldsymbol{n}$的阶乘(C++):} ]{codes/funfactorial.cpp} %C++

\lstinputlisting[ language=Java, title={\raggedright\normalsize 计算$\boldsymbol{n}$的阶乘(Java):} ]{codes/funfactorial.java} %Java

\lstinputlisting[ language=Python, title={\raggedright\normalsize 计算$\boldsymbol{n}$的阶乘(Python):} ]{codes/funfactorial.py} %Python

\lstinputlisting[style=Matlab-editor, title={\raggedright\normalsize 计算$\boldsymbol{n}$的阶乘(MATLAB):}]{codes/funfactorial.m} %MATLAB(如果不想要这种风格,则把该行命令的可选参数style=Matlab-editor改为language=Matlab

\end{appendices}
