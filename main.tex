% !Mode:: "TeX:UTF-8"

\documentclass{nemcmthesis}

% -------------------- 填写论文信息 -------------------- %
\ttle{这是论文题目} %【论文题目】(《通知》要求:论文题目直接用竞赛试题的标题,不要自行拟定.)
\category{研究生/本科生/专科生} %【参赛组别】
\school{某某某某大学} %【参赛学校】
\tihao{A/B/C/D} %【参赛题号】
\membera{队员1} %【队员1姓名】
\memberb{队员2} %【队员2姓名】
\memberc{队员3} %【队员3姓名】
\phonea{12377778900} %【队员1联系电话】
\phoneb{13477778900} %【队员2联系电话】
\phonec{14577778900} %【队员3联系电话】
\supervisor{教师组} %【指导教师姓名】
% ================================================== %


% ---------- 可以在此处添加:新加入的宏包、自定义的命令 ---------- %

% ============================================================ %


\begin{document}

\makecoverpage % 添加封面页. 注释该命令即可暂时删去封面页


% ---------- 摘要页 ---------- %

% ---------- 摘要页内容 ---------- %
\begin{abstract}

\zhlipsum[1-2]

\keywords{词建关1;词建关2;词建关3;词建关4;词建关5}

\end{abstract}

\newpage
% ================================================== %	


% ---------- 正文(划分为多个独立文件依次导入) ---------- %

\section{问题重述}

\subsection{问题背景}

\zhlipsum[1]

北冥有鱼,其名为鲲\upcite{书籍2024}。鲲之大,不知其几千里也。化而为鸟,其名为鹏\upcite{博士论文2024}。鹏之背,不知其几千里也;怒而飞,其翼若垂天之云。是鸟也,海运则将徙于南冥。南冥者,天池也\upcite{期刊2024}。

\zhlipsum[3]

\subsection{具体重述}

\zhlipsum[3]


 % 问题重述
\input{mainbody/ch2} % 问题分析
\input{mainbody/ch3} % 问题假设
\input{mainbody/ch4} % 符号说明
\input{mainbody/ch5} % 模型的建立与求解

\section{灵敏度分析}

\zhlipsum*[1]
 % 灵敏度分析
\input{mainbody/ch7} % 模型评价
\input{mainbody/ch8} % 型的改进与推广
\clearpage
% ================================================== %


% ---------- 参考文献 ---------- %
\bibliography{refs} %bib文件导入的形式生成参考文献列表
%\begin{thebibliography}{100}
%\bibitem{文献x标签}文献x信息
%\end{thebibliography}
\clearpage
% ================================================== %


% ---------- 附录 ---------- %

\begin{appendices}

\section{相关定理的证明}

\zhlipsum*[2]

\section{程序代码}

% 方式一:直接输入的形式插入代码

\begin{lstlisting}[language=C++,title={\raggedright 基础程序}]
    int main(){
        int i;
        printf("hello latex!\n");
        return 0;
    }
\end{lstlisting}

% 方式二:以文件形式插入代码

\lstinputlisting[ language=C++, title={\raggedright 计算$\boldsymbol{n}$的阶乘(C++):} ]{codes/funfactorial.cpp} %C++

\lstinputlisting[ language=Java, title={\raggedright 计算$\boldsymbol{n}$的阶乘(Java):} ]{codes/funfactorial.java} %Java

\lstinputlisting[ language=Python, title={\raggedright 计算$\boldsymbol{n}$的阶乘(Python):} ]{codes/funfactorial.py} %Python

\lstinputlisting[style=Matlab-editor, title={\raggedright 计算$\boldsymbol{n}$的阶乘(MATLAB):}]{codes/funfactorial.m} %MATLAB(如果不想要这种风格,则把该行命令的可选参数style=Matlab-editor改为language=Matlab

\end{appendices}

% ================================================== %


\end{document} 