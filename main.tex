% !Mode:: "TeX:UTF-8"

\documentclass{nemcmthesis}

% -------------------- 填写论文信息 -------------------- %
\ttle{这是论文题目} %【论文题目】(《通知》要求:论文题目直接用竞赛试题的标题,不要自行拟定.)
\category{研究生/本科生/专科生} %【参赛组别】
\school{某某某某大学} %【参赛学校】
\tihao{A/B/C/D} %【参赛题号】
\membera{队员1} %【队员1姓名】
\memberb{队员2} %【队员2姓名】
\memberc{队员3} %【队员3姓名】
\phonea{12377778900} %【队员1联系电话】
\phoneb{13477778900} %【队员2联系电话】
\phonec{14577778900} %【队员3联系电话】
\supervisor{教师组} %【指导教师姓名】
% ================================================== %


% ---------- 可以在此处添加:新加入的宏包、自定义的命令 ---------- %

% ============================================================ %


\begin{document}

\makecoverpage % 添加封面页. 注释该命令即可暂时删去封面页


% ---------- 摘要页 ---------- %

% ---------- 摘要页内容 ---------- %
\begin{abstract}

\zhlipsum[1-2]

\keywords{词建关1;词建关2;词建关3;词建关4;词建关5}

\end{abstract}

\newpage
% ================================================== %	


% ---------- 正文(划分为多个独立文件依次导入) ---------- %

\section{问题重述}

\subsection{问题背景}

\zhlipsum[1]

阿斯额腹部IE地方那是粉丝哦i人分包i,二发哈诶u饭不然女白哦额uFBA讹误i哦\upcite{书籍2024}人吧刚发阿斯恶妇白u哦额。的阿舒服巴斯夫哑巴翻一倍UI哦啊个儒雅我饿果\upcite{博士论文2024},然丫丫收入分布啊搜二附院噶笨鱼染发滚吧额我iUR发噶我饿,欧i染发司法部热搜乐然不规范搜饿啊日u研发八色哦啊上班你哦\upcite{期刊2024}IE如发吧色i哦染发吧诶哦让付,吧额UI染发哦奥i日本佛爱如何法瑞额发吧额UI污染合法。

\zhlipsum[3]

\subsection{具体重述}

\zhlipsum[3]


 % 问题重述

\section{问题分析}

基于文献\cite{会议2024}给出下列分析。%举例展示正文类型的参考文献引用

\zhlipsum[3]  % 问题分析

\section{问题假设}

\begin{enumerate}[label=(\arabic*)]
  \item{天空飘舞着五彩的泡沫,树木摇曳着无声的旋律。河流唱着混乱的曲调,山石讲述着无解的谜题。风中传来遥远的呼唤,却又瞬间消散在空气中。这是一个无意义的世界.}
  \item{天空飘舞着五彩的泡沫,树木摇曳着无声的旋律。河流唱着混乱的曲调,山石讲述着无解的谜题。风中传来遥远的呼唤,却又瞬间消散在空气中。这是一个无意义的世界.}
  \item{天空飘舞着五彩的泡沫,树木摇曳着无声的旋律。河流唱着混乱的曲调,山石讲述着无解的谜题。风中传来遥远的呼唤,却又瞬间消散在空气中。这是一个无意义的世界.}
  \item{天空飘舞着五彩的泡沫,树木摇曳着无声的旋律。河流唱着混乱的曲调,山石讲述着无解的谜题。风中传来遥远的呼唤,却又瞬间消散在空气中。这是一个无意义的世界.}
\end{enumerate} 


 % 问题假设

\section{符号说明}

% 跨页表格
\setlength{\tabcolsep}{5mm}{%设置列间距
\begin{longtable}{ccc} %可在此使用p{数值}设置列宽(注:p{...}默认居左表示)
    %\caption{The List of Notation} \label{tab:Notiations} \\
	\toprule
	{\bf 符号} & {\bf 解释} & {\bf 单位}\\
	\midrule
    \endfirsthead %以上是首页表头
    \midrule
    {\bf 符号} & {\bf 解释} & {\bf 单位}\\
    \midrule
    \endhead %以上是通用表头
    \midrule
    \endfoot %以上是通用表尾
    \bottomrule
    \endlastfoot %以上是末页表尾,以下是表格正文
    符号1 & 解释1 & 单位1 \\
    符号2 & 解释2 &  \\
    符号1 & 解释1 & 单位3 \\
\end{longtable}}{}  % 符号说明

\section{模型的建立与求解}

% -------------------- 问题 1 -------------------- %

\subsection{对问题1的求解}

\zhlipsum[1]

\begin{equation}\label{equ:麦克斯韦方程组}
\begin{cases}
\oint_{\bk{S}}\bk{D}\cdot\diff\bk{S} &= q_0, \\
\oint_{\bk{S}}\bk{B}\cdot\diff\bk{S} &= 0, \\
\oint_{\bk{l}}\bk{E}\cdot\diff\bk{l} &= -\iint_{\bk{S}}\frac{\partial \bk{B}}{\partial t}\cdot\diff\bk{S}, \\
\oint_{\bk{l}}\bk{H}\cdot\diff\bk{l} &= \bk{I}_0 + \iint_{\bk{S}}\frac{\partial \bk{D}}{\partial t}\cdot\diff\bk{S}
\end{cases}
\end{equation}

\zhlipsum[2]

\subsubsection{结果求解与分析}

%\fourthsection{四级标题名称} %必要情况下可能会使用四级标题

% -------------------- 问题 2 -------------------- %

\subsection{对问题2的求解}

\begin{gather}\label{equ:规划模型}
\min\quad{f(\bk{x})} \\
\text{s.t.}\quad\begin{cases}
    a \leq x_i\leq b &, j=1,2,\cdots ,\alpha \\
	\max\limits_{j=1,2,\cdots ,\alpha}{\{A_{i,j}\}} \leq 24 &, i=1,2,\cdots ,\beta_1\\
	\max\limits_{j=1,2,\cdots ,\alpha}{\{B_{i,j}\}} \leqslant 24 &, i=\beta_1, \beta_1+1,\cdots, \beta_2 \\
    \end{cases}
\end{gather}


\subsubsection{结果求解与分析}

% -------------------- 问题 3 -------------------- %

\subsection{对问题3的求解}

\zhlipsum[3]

\begin{equation}\label{equ:VandermondeDet}
D_n = \begin{pmatrix}
1 & 1 & \ldots & 1 \\
x_1 & x_2 & \ldots & x_n \\
\vdots & \vdots & & \vdots \\
x_1^{n-1} & x_2^{n-1} & \ldots & x_n^{n-1}
\end{pmatrix} = \prod_{1\leq j<i\leq n}{(x_i-x_j)}.
\end{equation}

\zhlipsum[4]

\subsubsection{结果求解与分析}

% -------------------- 问题 4 -------------------- %

\subsection{对问题4的求解}

\subsubsection{结果求解与分析}



 % 模型的建立与求解

\section{灵敏度分析}

{\bf 该章用于展示基于\verb|changes|宏包的批注功能。}

北冥有鱼,其名为鲲。鲲之大,不知其几千里也;化而为鸟,其名为鹏。鹏之背,不知其几千里也;怒而飞,其翼若垂天之云。\added[id=队员1, comment={少了一句}]{是鸟也,海运则将徙于南冥。}南冥者,天池也。《齐谐》者,志怪者也。《谐》之言曰:“鹏之徙于南冥也,水击三千里,抟扶摇而上者九万里,去以六月息者也。”野马也,尘埃也,生物之以息相吹也。\deleted[id=队员3, comment={这句话删掉}]{之乎者也。}天之苍苍,其正色邪?其远而无所至极邪?其视下也,亦若是则已矣。且夫水之积也不厚,则其负大舟也无力。覆杯水于\replaced[id=队员2, comment={用错词}]{坳堂}{水堂}之上,则芥为之舟,置杯焉则胶,水浅而舟大也。风之积也不厚,则其负大翼也无力。故九万里,则风斯在下矣,而后乃今培风;背负青天,而莫之夭阏者,而后乃今将图南。蜩与学鸠笑之曰:“我决起而飞,抢榆枋而止,时则不至,而控于地而已矣,奚以之九万里而南为?”适莽苍者,三餐而反,腹犹果然;适百里者,宿舂粮;适千里者,三月聚粮。之二虫又何知!

小知不及大知,小年不及大年。奚以知其然也?朝菌不知晦朔,蟪蛄不知春秋,此小年也\deleted[id=队员2]{,不亦乐乎}。楚之南有冥灵者,以五百岁为春,五百岁为秋;上古有大椿者,以八千岁为春,八千岁为秋,此大年也。而彭祖乃今以久特闻,众人匹之,不亦悲乎!汤之问棘也是已。穷发之北,有冥海者,天池也。有鱼焉,其广数千里,未有知其修者,其名为鲲。有鸟焉,其名为鹏,背若泰山,翼若垂天之云,抟扶摇羊角而上者九万里,绝云气,负青天,然后图南,且适南冥也。斥鴳笑之曰:“彼且奚适也?我腾跃而上,不过数仞而下,翱翔蓬蒿之间,此亦飞之至也。而彼且奚适也?”此小大之辩也。

\highlight[id=队员2]{故夫知效一官,行比一乡,德合一君,}而征一国者,其自视也,亦若此矣。而宋荣子犹然笑之。且举世誉之而不加劝,举世非之而不加沮,定乎内外之分,辩乎荣辱之境,斯已矣。彼其于世,未数数然也。虽然,犹有未树也。夫列子御风而行,泠然善也,旬有五日而后反。彼于致福者,未数数然也。此虽免乎行,犹有所待者也。\highlight[id=队员2, comment={建议对其展开分析}]{若夫乘天地之正,而御六气之辩,以游无穷者,彼且恶乎待哉}?故曰:至人无己,神人无功,圣人无名\comment[id=队员2]{建议把落款补上}。

\iffalse
只能用这个命令
来实现多行注释
\fi

 % 灵敏度分析

\section{模型评价}

\subsection{模型优点}

\begin{enumerate}[label=(\arabic*)]
  \item{天空飘舞着五彩的泡沫,树木摇曳着无声的旋律。河流唱着混乱的曲调,山石讲述着无解的谜题。风中传来遥远的呼唤,却又瞬间消散在空气中。这是一个无意义的世界.}
  \item{天空飘舞着五彩的泡沫,树木摇曳着无声的旋律。河流唱着混乱的曲调,山石讲述着无解的谜题。风中传来遥远的呼唤,却又瞬间消散在空气中。这是一个无意义的世界.}
  \item{天空飘舞着五彩的泡沫,树木摇曳着无声的旋律。河流唱着混乱的曲调,山石讲述着无解的谜题。风中传来遥远的呼唤,却又瞬间消散在空气中。这是一个无意义的世界.}
\end{enumerate}

\subsection{模型不足}

\begin{enumerate}[label=(\arabic*)]
  \item{天空飘舞着五彩的泡沫,树木摇曳着无声的旋律。河流唱着混乱的曲调,山石讲述着无解的谜题。风中传来遥远的呼唤,却又瞬间消散在空气中。这是一个无意义的世界.}
  \item{天空飘舞着五彩的泡沫,树木摇曳着无声的旋律。河流唱着混乱的曲调,山石讲述着无解的谜题。风中传来遥远的呼唤,却又瞬间消散在空气中。这是一个无意义的世界.}
  \item{天空飘舞着五彩的泡沫,树木摇曳着无声的旋律。河流唱着混乱的曲调,山石讲述着无解的谜题。风中传来遥远的呼唤,却又瞬间消散在空气中。这是一个无意义的世界.}
\end{enumerate}


 % 模型评价

\section{模型的改进与推广}

\subsection{模型改进}

\zhlipsum[1]

\subsection{模型推广}

\zhlipsum[2]


 % 型的改进与推广
\clearpage
% ================================================== %


% ---------- 参考文献 ---------- %
\bibliography{refs} %bib文件导入的形式生成参考文献列表
%\begin{thebibliography}{100}
%\bibitem{文献x标签}文献x信息
%\end{thebibliography}
\clearpage
% ================================================== %


% ---------- 附录 ---------- %

\begin{appendices}

\section{相关定理的证明}

\zhlipsum*[2]

\section{程序代码}

% 方式一:直接输入的形式插入代码

\begin{lstlisting}[language=C++,title={\raggedright 基础程序}]
    int main(){
        int i;
        printf("hello latex!\n");
        return 0;
    }
\end{lstlisting}

% 方式二:以文件形式插入代码

\lstinputlisting[ language=C++, title={\raggedright 计算$\boldsymbol{n}$的阶乘(C++):} ]{codes/funfactorial.cpp} %C++

\lstinputlisting[ language=Java, title={\raggedright 计算$\boldsymbol{n}$的阶乘(Java):} ]{codes/funfactorial.java} %Java

\lstinputlisting[ language=Python, title={\raggedright 计算$\boldsymbol{n}$的阶乘(Python):} ]{codes/funfactorial.py} %Python

\lstinputlisting[style=Matlab-editor, title={\raggedright 计算$\boldsymbol{n}$的阶乘(MATLAB):}]{codes/funfactorial.m} %MATLAB(如果不想要这种风格,则把该行命令的可选参数style=Matlab-editor改为language=Matlab

\end{appendices}

% ================================================== %


\end{document} 